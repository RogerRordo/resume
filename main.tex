% !TeX TS-program = xelatex

\documentclass{resume}
\ResumeName{罗卓枫}

% 如果想插入照片,请使用以下两个库。
% \usepackage{graphicx}
% \usepackage{tikz}

\begin{document}

\ResumeContacts{
  (+86)137-9036-2333,%
  \ResumeUrl{mailto:luozhf3@mail2.sysu.edu.cn}{luozhf3@mail2.sysu.edu.cn},%
  \ResumeUrl{https://github.com/RogerRordo}{github.com/RogerRordo},%
  \ResumeUrl{https://rordo.top}{rordo.top}%
}

% 如果想插入照片,请取消此代码的注释。
% 但是默认不推荐插入照片,因为这不是简历的重点。
% 如果默认的照片插入格式不能满足你的需求,你可以尝试调整照片的大小,或者使用其他的插入照片的方法。
% 不然,也可以先渲染 PDF 简历,然后用其他工具在 PDF 上叠加照片。
% \begin{tikzpicture}[remember picture, overlay]
%   \node [anchor=north east, inner sep=1cm]  at (current page.north east) 
%      {\includegraphics[width=2cm]{image.png}};
% \end{tikzpicture}

\ResumeTitle


\section{简介}

\textnormal{4年全栈开发经验。Pony.AI系统与工具链团队的软件工程师,负责自动驾驶数据处理与可视化平台的构建、优化和维护,为业务团队提供高效、可靠的数据支撑。}
\begin{itemize}
  \item 精通 Python 开发,熟悉 C++,擅长构建基于 Flask 与 Sqlalchemy ORM 的高性能 Web 应用,实现复杂数据处理。
  \item 深入掌握 Greenplum/PostgreSQL 和 TiDB/MySQL,具备丰富的数据库调优及大数据分析实践。
  \item 熟悉现代 DevOps 工具(Bazel、Git、Docker、Argo Workflow、Airflow、ELK),能快速搭建并持续优化容器化微服务。
  \item 了解 TypeScript、React、Redis、DBT、Spark、Iceberg,具备前后端协同开发和跨语言系统集成能力。
\end{itemize}

\section{工作经历}

\ResumeItem{广州小马智行 Pony.AI}
[\textnormal{系统与工具链部|} 软件开发工程师]
[2021.06—present]

\begin{itemize}
  \item \textbf{路测数据处理与可视化平台开发}
    \begin{itemize}
        \item 开发数据分析处理模块,处理 150TB/天 的车辆路测数据,实现运营时长、里程、订单、信号质量、车流/人流等指标的定制化采集和交付。
        \item 基于 Flask+React 框架,开发高度灵活的数据可视化平台。支持用户按任意维度聚合数据,并以多种图表及地图形式展示,满足各业务部门查询需求。
        \item 研发通用指标计算模块。通过 Jsonnet 定义计算逻辑和数据血缘,实现 95\%指标的低代码配置,降低业务团队使用门槛。
        \item 重构数据可视化平台的持续集成和测试体系。设计正交化的测试用例,建立分层测试框架(单元测试、基于 Cypress 的端到端测试、API 测试),将测试效率提升 7 倍以上。
    \end{itemize}
  \item \textbf{自动驾驶数据库开发维护与架构优化}
    \begin{itemize}
        \item 主导 ADB-PG(Greenplum) 分布式数据库大版本升级。设计并实施双写过渡方案,实现平稳迁移,使指标查询的 P95 延迟从 25s 降至 0.9s,超时率从 10.5‰ 降至 0.24‰。
        \item 建立数据库开发规范体系,输出《ADB-PG SQL 调优最佳实践》等技术文档,推动团队 SQL Review 规范落地。在对业务 SQL 和数据库设计的持续优化下,数据库 QPS 提升 44\%。
    \end{itemize}
  \item \textbf{商业化数据交付系统开发}
    \begin{itemize}
        \item 设计基于 Protobuf 的流数据交换协议,编写维护对外接口文档。开发自动化数据适配流水线(Argo Workflow),定时向合作方交付路测数据。
        \item 创新存储优化方案:通过点云数据扁平化存储、基于 Ramer-Douglas-Peucker 算法的轨迹降采样等技术手段,将存储成本降低 50\%。
    \end{itemize}
  \item \textbf{L2+车端核心模块开发}
    \begin{itemize}
        \item 主导设计、开发符合 ISO 26262 标准的L2+功能状态机架构,适配底盘、定位、感知、规控等模块接口,实现功能安全。
    \end{itemize}
\end{itemize}

\section{教育背景}

\ResumeItem
[中山大学|本科生]
{中山大学}
[\textnormal{光电科学技术与工程|} 工学学士]
[2016.09—2020.06]

\section{竞赛荣誉}

\ResumeItem
{2017 中国大学生程序设计竞赛 (CCPC) 秦皇岛站}
[\textnormal{银奖}]
[2017.10]
\ResumeItem
{2018 广东省大学生程序设计竞赛 (GDCPC)}
[\textnormal{一等奖}]
[2018.05]
\ResumeItem
{2017 广东省大学生程序设计竞赛 (GDCPC)}
[\textnormal{一等奖}]
[2017.05]

\section{开源项目}

\ResumeItem[TenMinsHot]
{\ResumeUrl{https://github.com/RogerRordo/TenMinsHot}{TenMinsHot}}
[作者]
\begin{itemize}
  \item 全自动每日生成十分钟时事快讯视频并投稿B站。
  \item 爬虫爬取腾讯新闻,ChatGPT总结全文,微软TTS转音频,Pillow生成slide图,moviewpy库合成视频,调用B站API上传。
\end{itemize}


\end{document}
